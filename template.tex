%%%%%%%%%%%%%%%%%
%% This is The CV texmplate created using nextcv.cls, 
%%    and the upgraded version of the sample.tex file provided by the original author.
%% Written by LianTze Lim, Naveen Dharmathunga
%% ================================================================================
%% It may be distributed and/or modified under the
%% conditions of the LaTeX Project Public License, either version 1.3
%% of this license or (at your option) any later version.
%% The latest version of this license is in http://www.latex-project.org/lppl.txt
%% and version 1.3 or later is part of all distributions of LaTeX
%% version 2003/12/01 or later.
%%%%%%%%%%%%%%%%

% Set PDF/A standard
\DocumentMetadata{pdfstandard=A-2a, pdfversion=1.7, lang=en}
%% Use the "normalphoto" option if you want a normal photo instead of cropped to a circle
% \documentclass[10pt,a4paper,normalphoto]{nextcv}
\documentclass[10pt,a4paper,ragged2e,withhyper]{nextcv}
%% NextCV uses the fontawesome5 and packages.
%% See http://texdoc.net/pkg/fontawesome5 for full list of symbols.

\setpdfmetadata{me}{mytitle}{subject}{Green Networking, Mobile Cloud, Network Coding, Energy}

% Change the page layout if you need to
\geometry{left=1.25cm,right=1.25cm,top=1.5cm,bottom=1.5cm,columnsep=1.2cm}

% The paracol package lets you typeset columns of text in parallel
\usepackage{paracol}

% Change the font if you want to, depending on whether you're using xelatex/lualatex
% WHEN COMPILING WITH XELATEX PLEASE USE
% xelatex -shell-escape -output-driver="xdvipdfmx -z 0" sample.tex
\setmainfont{Exo2}[
  Path           =./fonts/Exo_2/static/,
  Extension      =.ttf,
  UprightFont    =*-Regular,
  ItalicFont     =*-Italic,
  BoldFont       =*-Bold,
]
\setsansfont{Lato}
\renewcommand{\familydefault}{\sfdefault}

% Change the colours if you want to
\definecolor{SlateGrey}{HTML}{2E2E2E}
\definecolor{LightGrey}{HTML}{666666}
\definecolor{AstrosNavy}{HTML}{002D62}
\definecolor{EvertonBlue}{HTML}{003399}
\definecolor{Cerulean}{HTML}{2A52BE}
\definecolor{DodgerBlue}{HTML}{1E90FF}
\definecolor{UranianBlue}{HTML}{AFDBF5}

\colorlet{name}{black}
\colorlet{tagline}{EvertonBlue}
\colorlet{heading}{AstrosNavy}
\colorlet{headingrule}{UranianBlue}
\colorlet{subheading}{Cerulean}
\colorlet{accent}{EvertonBlue}
\colorlet{emphasis}{SlateGrey}
\colorlet{link}{DodgerBlue}
\colorlet{body}{LightGrey}

% Change some fonts, if necessary
\renewcommand{\namefont}{\Huge\rmfamily\bfseries}
\renewcommand{\personalinfofont}{\footnotesize}
\renewcommand{\cvsectionfont}{\LARGE\rmfamily\bfseries}
\renewcommand{\cvsubsectionfont}{\large\bfseries}


% Change the bullets for itemize and rating marker
% for \cvskill if you want to
\renewcommand{\cvItemMarker}{{\small\textbullet}}
% \renewcommand{\cvRatingMarker}{\faIcon{circle}}
% ...and the markers for the date/location for \cvevent
% \renewcommand{\cvDateMarker}{\faCalendar*[regular]}
% \renewcommand{\cvLocationMarker}{\faMapMarker*}


% If your CV/résumé is in a language other than English,
% then you probably want to change these so that when you
% copy-paste from the PDF or run pdftotext, the location
% and date marker icons for \cvevent will paste as correct
% translations. For example Spanish:
% \renewcommand{\locationname}{Ubicación}
% \renewcommand{\datename}{Fecha}


%% Use (and optionally edit if necessary) this .tex if you
%% want to use an author-year reference style like APA(6)
%% for your publication list
% \input{pubs-authoryear.cfg}

%% Use (and optionally edit if necessary) this .tex if you
%% want an originally numerical reference style like IEEE
%% for your publication list
%% \input{pubs-num.cfg}

%% sample.bib contains your publications
%% \addbibresource{sample.bib}

\begin{document}
%% ChkTeX will complain about wrong length of dashes. You can ignore it. or use {-} instead of -.
%% https://tex.stackexchange.com/a/687676

\name{Your Name Here}
\tagline{Your Position or Tagline Here}
%% You can add multiple photos on the left or right
\photoR{2.8cm}{Globe_High}

\personalinfo{%
  % Not all of these are required!
  \email{your_name@email.com}
  \phone{000{-}00{-}0000}
  \mailaddress{Address, Street, 00000 Country}
  % \location{Location, COUNTRY}
  \homepage{www.homepage.com}
  \linkedin{your_id}
  \github{your_id}
  \medium{@medium_handle}
  \twitter{xhandle}
  \orcid{0000--0000--0000--0000}
  %% You can add your own arbitrary detail with
  %% \printinfo{symbol}{detail}[optional hyperlink prefix]
  % \printinfo{\faPaw}{Hey ho!}[https://example.com/]
  %% Or you can declare your own field with
  %% \NewInfoFiled{fieldname}{symbol}[optional hyperlink prefix] and use it:
  % \NewInfoField{gitlab}{\faGitlab}[https://gitlab.com/]
  % \gitlab{your_id}
}

\makecvheader%
%% Depending on your tastes, you may want to make fonts of itemize environments slightly smaller
% \AtBeginEnvironment{itemize}{\small}

%% Set the left/right column width ratio to 6:4.
\columnratio{0.6}

% Start a 2-column paracol. Both the left and right columns will automatically
% break across pages if things get too long.
\begin{paracol}{2}
\cvsection{Experience}

\cvevent{Job Title 1}{Company}{Month 20XX --- Ongoing}{Location}
\cvsubproject{Your Project}{Your Role}{}{%
  \begin{itemize}%
  \item Project description 1
  \item Project description 2
  \end{itemize}%
}{C\#, {.NET}, SQLite}%

\cvsectionspace%


\cvsection{Education}

\cvevent{Ph.D.\ in Your Discipline}{Your University}{Sept 2002 --- June 2006}{Location}
Thesis title: Wonderful Research

\divider%

\cvevent{M.Sc.\ in Your Discipline}{Your University}{Sept 2001 --- June 2002}{Location}

\divider%

\cvevent{B.Sc.\ in Your Discipline}{Stanford University}{Sept 1998 --- June 2001}{Location}
Final Year Project: \textit{(Group Project)}\\
\begin{itemize}
\item \textbf{Title:} \textit{Project Title}
\item \textbf{Role:} \textit{Your role}
\item \textbf{Grade:} \textit{A+}
\end{itemize}

\cvsectionspace%

% use ONLY \newpage if you want to force a page break for
% ONLY the current column
\newpage

\cvsection{Projects}

\cvproject{Name 1, Customer experience optimizer and sales management system}
{Your Role}{https://github.com/you/your-repository}{%
  \begin{itemize}%
    \item Project description 1
    \item Project description 2
  \end{itemize}%
}{ASP.NET, Vue.js, Node.js, MS SQL}%

\divider%

\cvproject{Name 2, Project title in descriptive format}
{}{https://github.com/you/your-repository}{%
  \begin{itemize}%
    \item Project description 1
    \item Project description 2
  \end{itemize}%
}{Java, React.js}%

\cvsectionspace%


\cvsection{Open-Source Contributions}

\cvproject{Name1, Project title in descriptive format}
{}{https://github.com/you/your-repository}{%
  \begin{itemize}%
    \item Project description 1
    \item Project description 2
  \end{itemize}%
}{Java, React.js}%

\divider%

\cvproject{Name2, Project title in descriptive format}
{}{https://github.com/you/your-repository}{%
  \begin{itemize}%
    \item Project description 1
    \item Project description 2
  \end{itemize}%
}{Java, React.js}%

%%% ==============================================================================================================
%% Switch to the right column. This will now automatically move to the second
%% page if the content is too long.
\switchcolumn%

\cvsection{Career Objective}

A brief statement (2--3 sentences) highlighting your key qualifications, experience, and career goals.

\cvsectionspace%


\cvsection{Skills}

\cvtagheader{Cloud Technologies}
\cvskill{Azure}{4.5} %% Supports X.5 values.
\cvtag{Virtual Machines}
\cvtag{Azure DevOps}\\
\cvtag{Azure Cosmos DB}

\bigskip%

\cvskill{AWS}{3} %% Supports X.5 values.
\cvtag{EC2}
\cvtag{RDS}

\bigskip%

\cvskill{GCP}{4} %% Supports X.5 values.
\cvtag{BigQuery}
\cvtag{Dataproc}
\cvtag{Databrics}

\divider\smallskip%

\cvtagheader{Languages}
\cvtag{Python}
\cvtag{C\#}
\cvtag{SQL}
\cvtag{Java}\\
\cvtag{Javascript}
\cvtag{C++}

\divider\smallskip

\cvtagheader{Frameworks \& Libraries}
\cvtag{Apache Spark}
\cvtag{Pandas}
\cvtag{NumPy}\\
\cvtag{.NET}
\cvtag{ASP.NET}
\cvtag{Entity Framework}\\
\cvtag{Microsoft Community Toolkit}
\cvtag{WinUI}\\
\cvtag{Spring-boot}
\cvtag{Vue.js}
\cvtag{Unity}

\divider\smallskip

% ONLY the current column
% \newpage

\cvtagheader{Databases}
\cvtag{PostgreSQL}
\cvtag{MySQL}\\
\cvtag{MS SQL Server}

\divider\smallskip

\cvtagheader{Big Data Technologies}
\cvtag{Databricks}
\cvtag{Hive}
\cvtag{Airflow}

\divider\smallskip

\cvtagheader{VCS}
\cvtag{Git}
\cvtag{GitHub}
\cvtag{Bitbucket}\\
\cvtag{Azure DevOps}

\divider\smallskip

\cvtagheader{Tools}
\cvtag{Jupyter Notebooks}
\cvtag{PyCharm}\\
\cvtag{Visual Studio \& Code}
\cvtag{Docker}
\cvtag{Jira}
\cvtag{Confluence}

\cvsectionspace%


\cvsection{Certifications}

\cvcert{Certificate 1}{https://certificate/link/1}{Issuing Organization}

\medskip%

\cvcert{Certificate 2}{https://certificate/link/2}{Issuing Organization}

\cvsectionspace%


\cvsection{Archivements}

\cvachievement{\textbf{\faIcon{trophy}}}{Fantastic Achievement}{and some details about it}

\divider%

\cvachievement{\textbf{\faIcon{heart-pulse}}}{Another achievement}{more details about it of course}

\divider%

\cvachievement{\textbf{\faIcon{heart-pulse}}}{Another achievement}{more details about it of course}


%% Yeah I didn't spend too much time making all the
%% spacing consistent... sorry. Use \smallskip, \medskip,
%% \bigskip, \vspace etc to make adjustments.
\cvsectionspace%


\cvsection{Referees}

\cvref{Prof.\ Alpha Beta}{Role}{Institute}{a.beta@university.edu}
{Address Line 1\\Address line 2}

\divider%

\cvref{Prof.\ Gamma Delta}{Role}{Institute}{g.delta@university.edu}
{Address Line 1\\Address line 2}


\end{paracol}


\end{document}
